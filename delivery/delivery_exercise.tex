\documentclass[11pt, oneside]{article}   	% use "amsart" instead of "article" for AMSLaTeX format
\usepackage{geometry}                		% See geometry.pdf to learn the layout options. There are lots.
\geometry{letterpaper}                   		% ... or a4paper or a5paper or ... 
%\geometry{landscape}                		% Activate for rotated page geometry
%\usepackage[parfill]{parskip}    		% Activate to begin paragraphs with an empty line rather than an indent
\usepackage{graphicx}				% Use pdf, png, jpg, or eps§ with pdflatex; use eps in DVI mode
								% TeX will automatically convert eps --> pdf in pdflatex		
\usepackage{amsmath}

%SetFonts

%SetFonts


\title{Delivery Exercise A: UNIK4150}
\author{Rikesh Chauhan}
%\date{}							% Activate to display a given date or no date

\begin{document}
\maketitle

\section*{1 a)}
$EIRP$ stands for effective isotropic radiated power. It is the measure of how much power is radiated by an antenna equally in all direction in reference with an isotropic antenna. For a transmitter it is denoted as $P_{TI}$,  known as effective isotropic transmit power and for a receiver it is denoted as $P_{RI}$, known as effective isotropic received power. In case of the transmitter, it is given by
\begin{equation*}
EIPR =  P_{TI}=\frac{P_TG_T}{L_T}
\end{equation*}
And in case of the receiver, it is given by
\begin{equation*}
EIPR = P_{RI} = \frac{P_RG_R}{L_R}
\end{equation*}
where:
\begin{flalign*}
P_T /P_R &= \text {transmitted/received power} &\\
G_T/G_R&= \text {gain of transmitter/receiver} &\\
L_T/L_R&= \text {feeder loss of transmitter/receiver} &
\end{flalign*}



\section*{1 b)}
Some of the important characteristics of antenna are as follows:
\subsection*{Radiation intensity}
\subsection*{Directivity}
\subsection*{Effeciency}
\subsection*{Power gain}

\section*{1 c)}
\section*{1 d)}
\section{}
%\subsection{}


\end{document}  